\begin{deluxetable*}{lrll}
\tablecaption{Key assumptions and model parameters used in our simulation setup}
\tablehead{\colhead{Parameter} & \colhead{Value} & \colhead{Unit} & \colhead{Description}}
\startdata
\textbf{Stellar sample} &  &  &  \\
$G_\mathrm{max}$ & 16 &  & Maximum Gaia magnitude \\
$M_\mathrm{\star, max}$ & 1.5 & $M_\odot$ & Maximum stellar mass \\
Luminosity evolution &  &  & \citet{Baraffe1998} \\
~\\ \textbf{Planetary parameters} &  &  &  \\
$M_\mathrm{P}$ & 0.1 -- 2.0 & $M_\oplus$ & Planetary mass range \\
$R_\mathrm{P, min}$ & 0.75 & $R_\oplus$ & Minimum planet radius \\
Baseline mass–radius relation &  &  & \citet{Zeng2016} \SI{100}{\percent} $\mathrm{MgSiO_3}$\tablenotemark{a} \\
$\delta_\mathrm{min}$ & 80 & ppm & Minimum transit depth \\
$P_\mathrm{max}$ & 500 & day & Maximum orbital period [day] \\
$S$ & 10 -- 2000 & \SI{}{\watt\per\meter\squared} & Net instellation range \\
$S_\mathrm{thresh}$ & 280 & \SI{}{\watt\per\meter\squared} & Threshold instellation for runaway greenhouse \\
~\\ \textbf{Runaway greenhouse model} &  &  &  \\
Runaway greenhouse atmospheric models &  &  & \citet{Turbet2020,Dorn2021} \\
$x_{H_2O}$ & \SIrange{e-5}{0.1}{} &  & \rev{Bulk water} mass fraction (fiducial case: 0.005) \\
$f_\mathrm{rgh}$ & \SIrange{0}{1}{} &  & Dilution factor (fiducial case: 0.8) \\
~\\ \textbf{Priors} &  &  &  \\
$\Pi(S_\mathrm{thresh}$) & [10, 1000] & \SI{}{\watt\per\meter\squared} & Uniform \\
$\Pi(x_{H_2O})$ & [\SI{e-5}{}, \SI{0.1}{}] &  & Log-uniform \\
$\Pi(f_\mathrm{rgh})$ & [0, 1] &  & Uniform \\
$\Pi(\langle R_\mathrm{P}\rangle_\mathrm{out})$ & [0, 15] & $R_\oplus$ & Mean radius of non-runaway planets, uniform
\enddata
\tablenotetext{a}{For a comparison with alternative interior compositions, see Appendix~\ref{app:MR_relation}.}
\label{tab:params_table}
\end{deluxetable*}
