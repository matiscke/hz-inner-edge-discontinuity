% Define document class
\documentclass[twocolumn]{aastex631}

% Some entries inspired from a preamble by Adrian Price-Whelan, https://github.com/adrn/PhaseSpiralAsteroseismology/blob/main/tex/preamble.tex

\usepackage{showyourwork}

% Latex imports
\let\tablenum\relax             % necessary for AASTeX
\usepackage{siunitx}
\sisetup{range-phrase=-, range-units=single, separate-uncertainty=true}
\sisetup{separate-uncertainty=true}
\usepackage{blindtext}          % Filler text
\usepackage{xcolor}

% paper comments
\usepackage{comment}						 % comments that can be switched visible/invisible
\includecomment{comment}
%\specialcomment{outtake}{\begingroup\sffamily\color{gray}}{\endgroup}
\specialcomment{note}{\begingroup\sffamily\color{red!40!green!70!blue!90}}{\endgroup}
%\excludecomment{note}                       % switch notes off

%% switch TODO notes on/off
\usepackage[backgroundcolor=red!20!green!40!blue!10, textsize=tiny]{todonotes}
\usepackage{regexpatch}
\makeatletter
\xpatchcmd{\@todo}{\setkeys{todonotes}{#1}}{\setkeys{todonotes}{inline,#1}}{}{}
\makeatother
%\usepackage[disable]{todonotes}			% switches all todo notes to invisible

% ---------------------------------
% PAPER VARIABLES
\newcommand{\Nplanets}{\ensuremath{733}}
\newcommand{\percentageTransiting}{999}
\newcommand{\dmax}{\ensuremath{50\,\mathrm{pc}}}
\newcommand{\wrr}{0.001}


% ---------------------------------
% CONSTANTS/MISSIONS/ABBREVIATIONS

% SIunitx definitions
\DeclareSIUnit\mSun{M_\odot}
\DeclareSIUnit\Msun{M_\odot}
\DeclareSIUnit\mStar{M_\star}
\DeclareSIUnit\Mstar{M_\star}
\DeclareSIUnit\mEarth{M_\oplus}
\DeclareSIUnit\Mearth{M_\oplus}
\DeclareSIUnit\rEarth{R_\oplus}
\DeclareSIUnit\Rearth{R_\oplus}
\DeclareSIUnit\year{yr}
\DeclareSIUnit\au{au}
\DeclareSIUnit\dex{dex}
\DeclareSIUnit\ppm{ppm}
\DeclareSIUnit\eV{eV}

% Missions/Projects/Packages
\newcommand{\project}[1]{\textsl{#1}}
\newcommand{\rst}{\project{Nancy Grace Roman Space Telescope}}
\newcommand{\plato}{\project{PLATO}}
\newcommand{\cheops}{\project{CHEOPS}}
\newcommand{\kepler}{\project{Kepler}}
\newcommand{\emcee}{\project{emcee}}

% Stats / probability
\newcommand{\given}{\,|\,}
\newcommand{\norm}{\mathcal{N}}
\newcommand{\pdf}{\textsl{pdf}}

% Maths
\newcommand{\dd}{\mathrm{d}}
\newcommand{\transpose}[1]{{#1}^{\mathsf{T}}}
\newcommand{\inverse}[1]{{#1}^{-1}}
\newcommand{\argmin}{\operatornamewithlimits{argmin}}
\newcommand{\mean}[1]{\left< #1 \right>}

% Non-scalar variables
\renewcommand{\vec}[1]{\ensuremath{\bs{#1}}}
\newcommand{\mat}[1]{\ensuremath{\mathbf{#1}}}

% Unit shortcuts
\newcommand{\msun}{\ensuremath{\mathrm{M}_\odot}}
\newcommand{\mjup}{\ensuremath{\mathrm{M}_{\mathrm{J}}}}
\newcommand{\kms}{\ensuremath{\mathrm{km}~\mathrm{s}^{-1}}}
\newcommand{\mps}{\ensuremath{\mathrm{m}~\mathrm{s}^{-1}}}
\newcommand{\pc}{\ensuremath{\mathrm{pc}}}
\newcommand{\kpc}{\ensuremath{\mathrm{kpc}}}
\newcommand{\kmskpc}{\ensuremath{\mathrm{km}~\mathrm{s}^{-1}~\mathrm{kpc}^{-1}}}
\newcommand{\dayd}{\ensuremath{\mathrm{d}}}
\newcommand{\yr}{\ensuremath{\mathrm{yr}}}
\newcommand{\AU}{\ensuremath{\mathrm{AU}}}
\newcommand{\Kel}{\ensuremath{\mathrm{K}}}

% Misc.
\newcommand{\bs}[1]{\boldsymbol{#1}}

% Astronomy
\newcommand{\feh}{\ensuremath{{[{\rm Fe}/{\rm H}]}}}
\newcommand{\mh}{\ensuremath{{[{\rm M}/{\rm H}]}}}
\newcommand{\logg}{\ensuremath{\log g}}
\newcommand{\Teff}{\ensuremath{T_{\textrm{eff}}}}
\newcommand{\vsini}{\ensuremath{v\,\sin i}}
\newcommand{\gaia}{\textsl{Gaia}}

% Begin!
\begin{document}

% Title
\title{An open source scientific article about extrasolar magma oceans}

% Author list
\author[0000-0001-8355-2107]{Martin Schlecker}
\affiliation{Department of Astronomy/Steward Observatory, The University of Arizona, 933 North Cherry Avenue, Tucson, AZ 85721, USA}
\author{al.}


% Abstract
\begin{abstract}
    $\ldots$ magma oceans $\ldots$

    Here, we assess the ability of space and ground-based telescopes to test this hypothesis using Bioverse, a simulation framework that leverages contextual information from the overall planet population.
    We argue that in the near future, ESA's PLATO mission and NASA's Roman Space Telescope will be the most promising endeavors to constrain this demographic feature.
    For each of these missions, we identify the key mission design drivers that enable a statistically sound detection.
    We also show the unique synergy of these missions in answering this question, and what survey strategy optimizes the statistical power of the combined dataset.

    $\ldots$ its measurement will also provide insights into which stars harbor the nearest habitable worlds.
\end{abstract}

% Main body
\section{Introduction}
%\Blindtext[4]

\begin{figure*}
    \begin{centering}
        \includegraphics[width=\hsize]{figures/flowchart.pdf}
        \caption{Workflow of our hypothesis testing with Bioverse. In the first block, a stellar sample is generated based on XXX. The stars are then populated with planetary systems from XXX, to which a population-level trend can be applied. The planets' respective transit probabilities are computed. The second block simulates the exoplanet survey whereby selection effects and detection biases are introduced. Finally, the third block deals with testing a hypothesis based on the data from the simulated survey. By iterating through these steps, we compute the statistical power of testing the hypothesis given the assumed survey design.}
        \label{fig:flowchart}
    \end{centering}
\end{figure*}

%\begin{deluxetable*}{cc}
%\tablecaption{Confirmed and suggested trends in the exoplanet population}
%%\tablewidth{\pagewidth}
%\tablehead{\colhead{Trend} & \colhead{Reference}}
%\startdata
%missing sub-Saturn mass valley in M-dwarf planetary systems &  Schlecker et al., subm. \\
%$\cdots$ & $\cdots$
%\enddata
%\tablecomments{We show a list of population-level trends that have been reported in demographic studies or predicted from planet formation theories.}
%\end{deluxetable*}
%
%\begin{deluxetable*}{cccccc}
%\tablecaption{Confirmed and suggested trends in the exoplanet population}
%%\tablewidth{\pagewidth}
%\tablehead{\colhead{No.} & \colhead{Trend} & \colhead{Reference} & \colhead{Terrestrials}&\colhead{Roman}&\colhead{Plato}}
%\startdata
%%\hline
%%Theoretically predicted&&&&&
%%\\
%%\hline
%%1&High occurrence of low-mass planets at a few au&\citep{Penny2019}&FALSE&TRUE&FALSE \\
%%2&(super-)Earth occurrence drops again for very low-mass stars&Mulders2021, (Schlecker2021a)&TRUE&TRUE&TRUE
%%\\
%%3&Hot Jupiter - cold Jupiter relation&Schlaufman \& Winn (2016)&FALSE&TRUE&FALSE
%%\\
%%4&super-Earth composition-architecture link&Schlecker+2021a&TRUE&FALSE&FALSE
%%\\
%
%\enddata
%\tablecomments{We show a list of population-level trends that have been reported in demographic studies or predicted from planet formation theories.}
%\end{deluxetable*}



Several patterns in the planetary parameter space have been reported in demographic studies or predicted from planet formation theories.


\section{Synthetic star and planet sample}

\section{The XXX Hypothesis}

\section{Survey simulation}

\subsection{Testing the XXX hypothesis with the Nancy Grace Roman Telescope}



\section{Discussion}

\subsection{Mission design trades}\label{sec:mission-design-trades}
\citet{Penny2019} mention that a significant increase in planet yield could be achieved if the telescope's slew speed was increased.

\section{Conclusions}


\bibliographystyle{aasjournal}
\bibliography{bib}
\end{document}